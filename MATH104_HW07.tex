\documentclass[twoside,11pt]{article}
\usepackage[left=1in, right=1in, top=1in, bottom=1in]{geometry}
\usepackage{amsmath}
\usepackage{amssymb}
\usepackage{amsfonts}
\usepackage{mathtools}
\usepackage{amsthm}
\usepackage{fancyhdr}
\usepackage{enumitem}
\usepackage{siunitx}
\usepackage{booktabs}
\usepackage[hidelinks]{hyperref}
\usepackage{sectsty}
\usepackage{mathrsfs} % mathscr
\usepackage{tikz}
\usepackage{pgfplots}
\usepackage{multicol}
\usepackage{listings}

%define math operators %%%
\newcommand{\F}{\mathbb{F}}
\newcommand{\R}{\mathbb{R}}
\newcommand{\N}{\mathbb{N}}
\newcommand{\Z}{\mathbb{Z}}
\newcommand{\Q}{\mathbb{Q}}
\newcommand{\X}{\mathbb{Y}}
% \renewcommand{\d}{\mathrm{d}}
\renewcommand*\d{\mathop{}\!\mathrm{d}}
\DeclareMathOperator*{\argmax}{arg\,max}
\DeclareMathOperator*{\argmin}{arg\,min}
\DeclareMathOperator{\im}{im}
\DeclareMathOperator{\id}{id}

% section font style
\sectionfont{\sffamily\Large}
\subsectionfont{\sffamily\normalsize}
\subsubsectionfont{\bf}

% line spreading and break
\hyphenpenalty=5000
\tolerance=20
\setlength{\parindent}{0em}
\setlength\parskip{0.5em}
\allowdisplaybreaks
\linespread{0.9}

% theorem
% latex theorem
% definition style
\theoremstyle{definition}
\newtheorem{theorem}{Theorem}[subsection]
\newtheorem{axiom}{Axiom}[section]
\newtheorem{definition}{Definition}[section]
\newtheorem{example}{Example}[section]
\newtheorem{question}{Question}[section]
\newtheorem{exercise}{Exercise}[section]
\newtheorem*{exercise*}{Exercise}
\newtheorem{lemma}{Lemma}[section]
\newtheorem{proposition}{Proposition}[section]
\newtheorem{corollary}{Corollary}[section]
\newtheorem*{theorem*}{Theorem}
\newtheorem{problem}{Problem}
% remark style
\theoremstyle{remark}
\newtheorem*{remark}{Remark}
\newtheorem*{solution}{Solution}
\newtheorem*{claim}{Claim}


% paragraph indent
\setlength{\parindent}{0em}
\setlength\parskip{0.5em}

\newcommand\Code{MATH104 SU22}
\newcommand\Ass{HW\#7}
\newcommand\name{Haoran Sun}
\newcommand\mail{haoransun@berkeley.edu}

\title{{\sffamily \Code \ \Ass}}
\author{\sffamily \name \ (\href{mailto:haoransun@berkeley.edu}{haoransun@berkeley.edu})}
\date{\sffamily \today}

\makeatletter
% \let\Title\@title
\let\theauthor\@author
\let\thedate\@date

\fancypagestyle{plain}{%
    \fancyhf{}
    \lhead{\sffamily \Ass}
    \rhead{\sffamily \name}
    \rfoot{\sffamily\thepage}

    % # 页脚自定义
    \fancyfoot[L]{
        \begin{minipage}[c]{0.06\textwidth}
            \includegraphics[height=12mm]{primarylogo.png}
        \end{minipage}
    }
}
\fancypagestyle{title}{%
    \fancyhf{}
    \renewcommand{\headrulewidth}{0pt}
    % \lhead{\Title}
    % \rhead{\theauthor}
    \rfoot{\sffamily\thepage}

    % # 页脚自定义
    \fancyfoot[L]{
        \begin{minipage}[c]{0.06\textwidth}
            \includegraphics[height=12mm]{primarylogo.png}
        \end{minipage}
    }
}
\fancyfootoffset[L]{0.4cm}

% re-define title format
\makeatletter
\renewcommand{\maketitle}{\bgroup\setlength{\parindent}{0pt}
\begin{flushleft}
  \textbf{\Large\@title}

  \@author
\end{flushleft}\egroup
}
\makeatother

\pagestyle{plain}

% lstlisting settings
\lstset{
    basicstyle=\linespread{0.7}\footnotesize,
    breaklines=true,
    basewidth=0.5em
}


\begin{document}
\maketitle
\thispagestyle{title}

\begin{remark}
    $V_\epsilon(x)$ is used to denote a $\epsilon$-neighborhood
    \begin{align*}
        V_\epsilon(x) = B_\epsilon(x)\setminus\{x\}
    \end{align*}
\end{remark}

\begin{enumerate}

\item \begin{enumerate}
    \item To let $|x(1-x)|<1$, we can derive that $x(1-x)\leq 0.5$ and thus
    $x\in((1-\sqrt[]{5})/2, (1+\sqrt[]{5})/2)$.
    Thus, $f_n(x)$ converges pointwise in this open interval and 
    converges uniformly in any closed interval in this interval.

    \item $f_n(x)$ converges pointwise on $\R$, and it converges on $\{-1, 1\}$ and any
    closed interval in $(-\infty,-1)\cup (-1,1)\cup (1,\infty)$.

    \item $S_n(x)$ converges pointwise on $((1-\sqrt[]{5})/2, (1+\sqrt[]{5})/2)$,
    and converges uniformly on any closed interval in $((1-\sqrt[]{5})/2, (1+\sqrt[]{5})/2)$.
\end{enumerate}


\item Since $f_n\rightarrow f$ converges uniformly on $(a,b)\cap\Q$, 
we know $f$ is a function defined on $(a,b)\cap\Q$.
Extend the function $f$ as $g:[a, b]\rightarrow\R$ by
\begin{align*}
    g(x) = \begin{cases}
        f(x) & x\in(a,b)\cap \Q\\
        \lim_{r\rightarrow x}f(r), r\in\Q & \text{otherwise}
    \end{cases}
\end{align*}
\begin{claim}
    Limit 
    \begin{align*}
        \lim_{r\rightarrow x}f(r), r\in\Q
    \end{align*}
    exists.
\end{claim}
\begin{proof}
    Consider any sequence $a_n\rightarrow x$ where $a_n\in(a,b)\cap\Q$.
    $\forall\epsilon$, $\exists N\in\N$ s.t. $\forall n\geq N$ we have
    \begin{align*}
        |f_n(r) - f(r)|<\frac{\epsilon}{3}
    \end{align*}
    $\forall r\in(a, b)\cap\Q$.
    Then, since $f_N$ continuous on $[a,b]$, $\forall\epsilon>0$,
    $\exists\delta>0$ s.t.
    \begin{align*}
        |x-y|\in V_\delta(0)\Rightarrow
        |f_N(x)-f_N(y)|\leq\frac{\epsilon}{3}
    \end{align*}
    For this $\delta>0$, since $a_n$ Cauchy, we can find $N'\in\N$
    s.t. $\forall m,n>N'$, we have
    \begin{align*}
        |a_n-a_m|<\delta
    \end{align*}
    Therefore
    \begin{align*}
        |f_N(a_n) - f_N(a_m)|<\frac{\epsilon}{3}
    \end{align*}
    Thus $\forall\epsilon$, $\exists N,N'\in\N$ s.t.
    $\forall m,n>N'$ we have
    \begin{align*}
        |f(a_n)-f(a_m)| &\leq
        |f(a_n)-f_N(a_n)| + |f(a_m)-f_N(a_m)| + |f_N(a_n)-f_N(a_m)|\\
        &\leq \frac{\epsilon}{3} + \frac{\epsilon}{3} + \frac{\epsilon}{3} = \epsilon
    \end{align*}
    Thus we know that $\forall a_n\rightarrow x$ and $a_n\in\Q$, $f(a_n)$
    also Cauchy, then limit $f(a_n)$ exists.

    To show $f(a_n)$ converges to the unique value for any $a_n\rightarrow x$,
    the scheme is similar as above.
    Therefore
    \begin{align*}
        \lim_{r\rightarrow x}f(r) 
    \end{align*}
    exists.
\end{proof}
\begin{claim}
    $f_n\rightarrow g$ uniformly on $[a,b]$.
\end{claim}
\begin{proof}
    We already know that $f_n\rightarrow g$ uniformly on $(a, b)\cap \Q$.
    Now consider any $x\in [a,b]\setminus((a,b)\cap\Q)$.
    $\forall\epsilon$ $\exists N\in\N$ s.t. $\forall n\geq N$ we have
    \begin{align*}
        |f(r)-f_n(r)|<\frac{\epsilon}{3}
    \end{align*}
    $\forall r\in(a,b)\cap \Q$.
    Also, since $g(x)$ is defined as
    \begin{align*}
        g(x) = \lim_{r\rightarrow x}f(r)
    \end{align*}
    Then $\forall\epsilon$, $\exists\delta_1>0$ s.t.
    \begin{align*}
        r\in V_{\delta_1}(x), r\in(a,b)\cap\Q\Rightarrow
        |f(r) - g(x)|<\frac{\epsilon}{3}
    \end{align*}
    Also, by the continuity of $f_n$, $\forall\epsilon$,
    $\exists\delta_2>0$ s.t.
    \begin{align*}
        r\in V_{\delta_2}(x), r\in(a,b)\cap\Q\Rightarrow
        |f_n(x)-f_n(r)|<\frac{\epsilon}{3}
    \end{align*}
    Hence, $\forall\epsilon>0$, $\exists N\in\N$ s.t. $\forall n\geq N$,
    $\exists \delta_1,\delta_2>0$,
    $r\in V_{\min(\delta_1,\delta_2)}(x), r\in(a,b)\cap\Q$ we have
    \begin{align*}
        |f_n(x)-g(x)| &\leq |f_n(x)-f_n(r)| + |f_n(r)-f(r)| + |f(r)-g(x)|\\
        &<\frac{\epsilon}{3} + \frac{\epsilon}{3} + \frac{\epsilon}{3} = \epsilon
    \end{align*}
    Thus $f_n\rightarrow g$ uniformly.
\end{proof}
Therefore $f_n$ converges uniformly on $[a,b]$.


\item Let $|f|$ and $|g|$ bounded by $M>0$, then
$\forall\epsilon>0$ $\exists N\in\N$ s.t.
$\forall n\geq N$ we have
\begin{align*}
    |f_n(x)-f(x)|<\frac{\epsilon}{2M}\\
    |g_n(x)-g(x)|<\frac{\epsilon}{4M}
\end{align*}
Since $|f|$ bounded by $M$ and $f_n\rightarrow f$ uniformly,
$\exists N'\in\N$ s.t. $\forall n\geq N'$ $|f_n(x)|$ bounded by $2M$.
Thus, $\forall\epsilon>0$, $\exists N,N'\in\N$ s.t.
$\forall n\geq\max(N,N')$ we have
\begin{align*}
    |f_n(x)g_n(x)-f(x)g(x)|&\leq
    |f_n(x)||g_n(x)-g(x)| + |g(x)||f(x)-f_n(x)|\\
    &< 2M\frac{\epsilon}{4M} + M\frac{\epsilon}{2M} = \epsilon
\end{align*}
$\forall x\in\R$.
Thus $f_ng_n\rightarrow fg$ uniformly on $\R$.


\item Since
\begin{align*}
    f(x) = \frac{1}{1-x} = \sum_{n=0}^\infty x^n
\end{align*}
$\forall x\in (-1,1)$,
it is easy to prove that $f'_n\rightarrow f'$
and $f''_n\rightarrow f''$ and hence
\begin{align*}
    f''(x) = \frac{2}{(1-x)^3} = \sum_{n=1}^\infty n(n+1)x^{n-1}
\end{align*}
Thus we have
\begin{align*}
    \frac{2x}{(1-x)^3} = \sum_{n=1}^\infty n(n+1)x^n
\end{align*}
Substitute $x$ to $x-1$, we have
\begin{align*}
    \sum_{n=1}^\infty n(n+1)(x-1)^n = \frac{2(x-1)}{(2-x)^3}
\end{align*}
By root test, we have convergence radius $\rho = 1$, then
$f(x)$ converges on $(0, 2)$.


\item The Taylor expansion for $\ln(1+x)$ appears to be
\begin{align*}
    \ln(1+x) = \sum_{n=1}^\infty (-1)^{n-1}\frac{1}{n}x^n
\end{align*}
Thus
\begin{align*}
    \ln(1-1/3) = \sum_{n=1}^\infty (-1)^{n-1}\frac{3^{-n}}{n}
    = -\sum_{n=1}^\infty\frac{3^{-n}}{n}
\end{align*}
Therefore
\begin{align*}
    \sum_{k=1}^\infty\frac{3^{-k}}{k} = \ln\frac{3}{2}
\end{align*}


\item \begin{enumerate}
    \item Let $a_n$ decrease to $0$ and 
    \begin{align*}
        \sum_{k=1}^n b_k
    \end{align*}
    is bounded forall $n$.
    Then 
    \begin{align*}
        \sum a_kb_k
    \end{align*}
    converges.
    \item Use $f_n(x)$ to denote
    \begin{align*}
        \sum_{k=1}^n \frac{(-1)^k}{x^4+k}
    \end{align*}
    Since it is a alternative series and $1/(x^4+n)\rightarrow 0$
    as $n\rightarrow 0$, then it converges pointwisely $\forall x\in\R$.

    \begin{claim}
        $\forall\epsilon$, $\exists N\in\N$ s.t. $\forall n>m\geq\frac{1}{N}$,
        we have
        \begin{align*}
            |f_n(x) - f_m(x)| < \epsilon
        \end{align*}
        $\forall x\in\R$.
    \end{claim}
    \begin{proof}
        $\forall\epsilon>0$, we can pick $N\in\N$ s.t.
        $\forall n>m\geq N$, we have $1/N<\epsilon$ and
        \begin{align*}
            f_n(x)-f_m(x) &= \sum_{k=m+1}^n\frac{(-1)^k}{x^4+k}\\
            &= (-1)^(m+1)\left(
                \frac{1}{x^4+m+1} - \frac{1}{x^4+m+2} + \cdots
            \right)\\
            \Rightarrow |f_n(x) - f_m(x)| &= 
            \frac{1}{x^4+m+1} - \frac{1}{x^4+m+2} + \cdots\\
            &\leq \frac{1}{x^4+m+1}\leq \frac{1}{m+1}\\
            &<\frac{1}{N}<\epsilon
        \end{align*}
        $\forall x$.
    \end{proof}
    Therefore $f_n(x)$ converges uniformly on $\R$.

    However, $f_n(x)$ does not converge absolutely.
\end{enumerate}


\item Note that $f$ bounded on $[0,1]$ and  continuous on
$[c, 1]$ $\forall c\in(0,1)$, then $f$ integrable on any
$[c,1]$.
Now consider the integrability of $f$ on $[0,c]$.
Let $P$ be any partition of $[0,c]$, we have
\begin{align*}
    U(f,P) - L(f,P) &= 
    (\sup\{f(x)|x\in I_k\} - \inf\{f(x)|x\in I_k\})l_k\\
    &\leq 2\sqrt{a_k}l_k\leq 2c\sqrt{c}
\end{align*}
Thus, $\forall\epsilon>0$, $\exists c<(\epsilon/2)^{2/3}$ s.t.
\begin{align*}
    U(f,P) - L(f,P) < \epsilon
\end{align*}
which means $f$ integrable on $[0,c]$ ($c$ depends on $\epsilon$).
Thus, $f$ integrable on $[0,1]$ since it is both integrable on
$[0,c]$ and $[c,1]$.



\item Let $P = \{a_0, a_1, \dots, a_n\}$, then
\begin{align*}
    g(x) = g(a_k),\ x\in I_k
\end{align*}
Since 
\begin{align*}
    L(g,P) \leq L(g)\leq U(g)\leq U(g,P)
\end{align*}
we have
\begin{align*}
    L(g,P) = L(g) = U(g) = U(g,P) = \int_a^bg(x)\d x
\end{align*}


\item Let $P = \{0, 1/N, 2/N, \dots, 1\}$, 
$\forall\epsilon>0$, $\forall N>\{1,3/\epsilon\}$ we have
\begin{align*}
    U(f, P) - L(f, P) &= \sum_{i=1}^N
    \left[
        \left(\frac{i}{N}\right)^2 - 
        \left(\frac{(i-1)}{N}\right)^2
    \right]\frac{1}{N}\\
    &=\sum_{i=1}^N\frac{1}{N^3}(2i+1)\\
    &=\frac{N^2 + 2N}{N^3} < \frac{3N^2}{N^3}<\epsilon
\end{align*}
which shows that $f$ integrable.


\end{enumerate}


\end{document}

